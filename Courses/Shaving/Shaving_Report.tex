\documentclass[12pt,a4paper]{article}

\usepackage{amsfonts}
\usepackage{amsmath}
\usepackage{geometry}
\usepackage{xcolor,graphicx}
%\usepackage[subfolder,cleanup]{gnuplottex}
%\usepackage{amsthm}
%\usepackage{enumitem}
%\usepackage{wrapfig}
%\usepackage{subcaption}
%\usepackage{hyperref}
%\usepackage{tikz}

%\usepackage[table]{xcolor}
%\rowcolors{2}{gray!25}{white}

\let\originalleft\left
\let\originalright\right
\renewcommand{\left}{\mathopen{}\mathclose\bgroup\originalleft}
\renewcommand{\right}{\aftergroup\egroup\originalright}

\newcommand{\diff}[3][]{\frac{d^{#1}#2}{d{#3}^{#1}}}
\newcommand{\pdiff}[3][]{\frac{\partial^{#1}#2}{\partial{#3}^{#1}}}

\bibliographystyle{ieeetr}

\title{Shaving Challenge}
\author{Brady Metherall}
\date{14 October, 2019}

\newgeometry{margin=1in}
%\setlength\parindent{0pt}

\begin{document}
\maketitle

main paper,  \cite{fitt} \\


kinds of hair (anagen and telegen), prediciton of hair growth and hair removal methods, two models for hair growth \cite{kolinko} \\

model shape of hair using euler bernoulli for beam deflection \cite{howison} \\


\section{Effect of shaving on growth}
clipping stimusates hair growth is some species by incuding anagen, effect of shaving on hair production (legs), shaving does not stimutae growth or incrrease growth rate \cite{lynfield} \\
face shaving effect on beard growth, hairs cut at 45 angle, shaving had no effect on growth \cite{trotter} \\



\section{Food}
In addition to models specific to hair, there have been multiple attempts to model slicing and cutting more generally. Typically these models have been developed for industrial food production. These models take one of two approaches---derived from energy conservation \cite{gubenia, zhou} or from stress tensors \cite{zhou}. Furthermore, while cutting there are two axes of motion, pushing the blade downwards into the media and dragging the blade along the surface. Each of these motions is itself ineffective at cutting, and a optimal `push / slice' ratio exists \cite{atkins2004, atkins2005}. This notion also explains why straight razors usually do not pose a threat to a cut---the razor is moved perpendicular to the blade and thus, there is no slicing motion.




\newpage
\bibliography{Ref}

\end{document}







