\documentclass[11pt,a4paper]{article}

\usepackage{amsfonts}
\usepackage{amsmath}
\usepackage{geometry}
\usepackage{xcolor}
\usepackage{graphicx}
%\usepackage[subfolder,cleanup]{gnuplottex}
%\usepackage{amsthm}
%\usepackage{enumitem}
%\usepackage{wrapfig}
%\usepackage{subcaption}
%\usepackage{hyperref}
%\usepackage{tikz}

\usepackage{enumerate}
\usepackage[shortlabels]{enumitem}


\usepackage{lipsum}


% Nicer brackets for operators
\let\originalleft\left
\let\originalright\right
\renewcommand{\left}{\mathopen{}\mathclose\bgroup\originalleft}
\renewcommand{\right}{\aftergroup\egroup\originalright}

% Math operators
\providecommand{\bigO}[1]{\ensuremath{\mathop{}\mathopen{}\mathcal{O}\mathopen{}\left(#1\right)}}

% Macros
\newcommand{\diff}[3][]{\frac{\textrm{d}^{#1}#2}{\textrm{d}{#3}^{#1}}}
\newcommand{\pdiff}[3][]{\frac{\partial^{#1}#2}{\partial{#3}^{#1}}}
\newcommand{\df}{\, \textrm{d}}
%\newcommand{\eps}{\varepsilon}

% Row colouring in tables
%\usepackage[table]{xcolor}
%\rowcolors{2}{gray!25}{white}

% Margin size
\newgeometry{margin=2cm}

% Reference style
%\bibliographystyle{ieeetr}

\title{Optimization Assignment}
\author{Brady Metherall}
\date{16 December 2019}

\begin{document}
\maketitle


\textbf{Problem 1.}
The new system consist of
\begin{align}
    \sum_j^n a_{ij} x_j &\leq b_i, & i &\in M_0, \label{eq:m0} \\
    \sum_j^n (a_{ik} a_{lj} - a_{lk} a_{ij}) x_j, &\leq a_{ik} b_l - a_{lk} b_i & (i,l) &\in M_+ \times M_-. \label{eq:mplus}
\end{align}
The coefficient of $x_k$ in \eqref{eq:m0} and \eqref{eq:mplus} is 0 since $i \in M_0$ and $a_{ik} a_{lk} - a_{lk} a_{ik} = 0$, respectively. Therefore, $x_k$ is not in this new system.


\textbf{Problem 2.}
\begin{enumerate}[i)]
    \item
    If both systems had a solution, that would imply
    \begin{align*}
        \mathbf{0}^T &= yA, \\
        0 &= (y A) x, \\
        &= y (A x), \\
        &\leq yb, \\
        &< 0,
    \end{align*}
    which is a contradiction. Thus, both systems cannot have solutions.
    \item
    There must exist a $k_*$ such that $d_{k_*} < 0$, otherwise, our new system is in fact consistent.
\end{enumerate}

\textbf{Problem 3.}
\begin{enumerate}[i)]
    \item Both $y_t$ and $z_t$ are binary variables. If production occurs within period $t$, then $y_t = 1$. Furthermore, if production is switched on within period $t$, then $z_t = 1$.
    \item
    \begin{align}
        \begin{pmatrix}
            \mathbf{0}^T & \mathbf{1}^T \\
            I - L & -I \\
            -I & I \\
            I & 0 \\
            0 & I
        \end{pmatrix}
        \begin{pmatrix}
            y \\
            z
        \end{pmatrix}
        \leq
        \begin{pmatrix}
            k \\
            \mathbf{0} \\
            \mathbf{0} \\
            \mathbf{1} \\
            \mathbf{1}
        \end{pmatrix}
    \end{align}

    \begin{align}
        \begin{pmatrix}
            \mathbf{0} & I - L^T & -I \\
            \mathbf{1} & -I & I
        \end{pmatrix}
    \end{align}
\end{enumerate}

\textbf{Problem 4.}
\begin{enumerate}[i)]
    \item The constraint matrix is the vertex-edge incidence matrix, $A$. Therefore, each column contains exactly two 1s. We can then choose the partitions to be $M_1 = V_1$ and $M_2 = V_2$, and the constraint matrix is totally unimodular.
    \item The relaxation of the LP is
    \begin{align}
        \nonumber \max_x \sum_{e \in E} x_e, \\
        \text{s.t.} \quad A x &\leq 1, \label{eq:edges} \\
        \nonumber x_e &\geq 0.
    \end{align}
    The dual is then given by
    \begin{align}
        \nonumber \min_y \sum_{v \in V} y_v, \\
        \text{s.t} \quad A^T y \geq 1, \label{eq:nodes} \\
        \nonumber y_v \geq 0.
    \end{align}
    Since $A^T$ is still totally unimodular this solves the IP problem as well. Additionally, for the minimum cardinality node covering, we do not require $y_v$ to be greater than one, and so $y_v \in \{ 0, 1 \}$.
    \item We can find trivial feasible solutions to \eqref{eq:edges} and \eqref{eq:nodes} is the empty matching and the full graph. Then by the Strong Duality Theorem and since $A$ is totally unimodular, K\"{o}nig's Theorem holds.
\end{enumerate}

\textbf{Problem 5.}

\textbf{Problem 6.}

\textbf{Problem 7.}

\textbf{Problem 8.}

\textbf{Problem 9.}


%\bibliography{Ref}

\end{document}
