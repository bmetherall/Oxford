\documentclass{beamer}

\usepackage[subfolder,cleanup]{gnuplottex}
\usepackage{animate}

\usetheme{InFoMM}

\title{Predicting Online Sales}
\date{December 2019}
\author{Brady Metherall}

\begin{document}
\frame{\titlepage}

\begin{frame}
	\frametitle{Motivation}
	\begin{itemize}
		\item We have data of the number of online sales for 2000 days.
		\item Some days our competitors use additional marketing which effects our sales, but, we do not know which days.
		\item We model the distribution and use a computer to fit the model to the data.
	\end{itemize}
\end{frame}

\begin{frame}[fragile]
    \frametitle{Daily Sale Data}
    \begin{figure}
        \centering
        \begin{gnuplot}[terminal=epslatex, terminaloptions={color size 4.0in,2.67in lw 3}]
            set grid
            set xl 'Units Sold'
            set yl 'Number of Days'

            p '../SaleData.dat' using 1:(1.0) smooth frequency w boxes not
        \end{gnuplot}
    \end{figure}
\end{frame}

\frame{
    \frametitle{Model Solution}
    \begin{figure}
        \centering
        \animategraphics[loop, autoplay]{4}{../Animate/Frames/Frame}{0}{16}
    \end{figure}
}

\frame{
	\frametitle{Results}
	\begin{itemize}
		\item Our competitors use additional marketing approximately 50\% of the time.
		\item We sell an average of 50 units when they do not have additional marketing, and 20 units when they do.
		\item What if our data is more complicated?
	\end{itemize}
}

\begin{frame}[fragile]
    \frametitle{More Complicated Data}
    \begin{figure}
        \centering
        \animategraphics[loop, autoplay]{24}{../Animate/Frames/FakeFrame}{0}{145}
    \end{figure}
\end{frame}

\end{document}
