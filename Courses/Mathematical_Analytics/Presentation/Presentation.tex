\documentclass{beamer}

\usepackage[subfolder,cleanup]{gnuplottex}
\usepackage{animate}

\usetheme{InFoMM}

\title{}
\date{December 2019}
\author{Brady Metherall}

\begin{document}
\frame{\titlepage}

\begin{frame}[fragile]
    \frametitle{Sale Data}
    \begin{figure}
        \centering
        \begin{gnuplot}[terminal=epslatex, terminaloptions={color size 4.0in,2.67in lw 3}]
            set grid
            set xl 'Units Sold'
            set yl 'Number of Days'

            p '../SaleData.dat' using 1:(1.0) smooth frequency w boxes not
        \end{gnuplot}
    \end{figure}
\end{frame}



\begin{frame}[fragile]
    \frametitle{Sale Data}
    \begin{figure}
        \centering
        \begin{gnuplot}[terminal=epslatex, terminaloptions={color size 4.0in,2.67in lw 3}]
            set grid
            set xl 'Units Sold'
            set yl 'Number of Days'
            set samples 1000

            p '../SaleData.dat' using 1:(1.0) smooth frequency w boxes not, \
            '../Animate/Fit6.dat' u 1 smooth csplines w l lc 7 lw 1.5 not
        \end{gnuplot}
    \end{figure}
\end{frame}


\frame{
    \frametitle{Solution}
    \begin{figure}
        \centering
        \animategraphics[loop, autoplay]{4}{../Animate/Frames/Frame}{0}{16}
    \end{figure}
}


\begin{frame}[fragile]
    \frametitle{Pretend Sale Data}
    \begin{figure}
        \centering
        \animategraphics[loop, autoplay]{24}{../Animate/Frames/FakeFrame}{0}{145}
    \end{figure}
\end{frame}

\end{document}
