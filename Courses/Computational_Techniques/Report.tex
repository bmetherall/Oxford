\documentclass[11pt, a4paper, twocolumn]{article}

\usepackage{amsfonts}
\usepackage{amsmath}
\usepackage{geometry}
\usepackage{xcolor,graphicx}
\usepackage[subfolder,cleanup]{gnuplottex}
\usepackage{amsthm}
%\usepackage{enumitem}
%\usepackage{wrapfig}
\usepackage{subcaption}
%\usepackage{hyperref}
%\usepackage{tikz}
\usepackage{algorithmicx}
\usepackage{algpseudocode}
\usepackage{algorithm}

%\usepackage[table]{xcolor}
%\rowcolors{2}{gray!25}{white}

\let\originalleft\left
\let\originalright\right
\renewcommand{\left}{\mathopen{}\mathclose\bgroup\originalleft}
\renewcommand{\right}{\aftergroup\egroup\originalright}

\DeclareMathOperator{\argmin}{arg\,min}

\newcommand{\diff}[3][]{\frac{d^{#1}#2}{d{#3}^{#1}}}
\newcommand{\pdiff}[3][]{\frac{\partial^{#1}#2}{\partial{#3}^{#1}}}
\newcommand{\df}{\, \textrm{d}}

\newtheorem{theorem}{Theorem}

\bibliographystyle{ieeetr}

\title{Randomized SVD}
\author{Brady Metherall}
\date{25 November 2019}

\newgeometry{margin=2cm}
%\setlength\parindent{0pt}

\begin{document}
\maketitle

\begin{algorithm}
    \begin{algorithmic}
        \State Input: Matrix $A$ of size $m \times n$, Int $r$, Int $l$.
        \State Output: Matrix of size $m \times n$ of rank-$r$.
        \State $\Omega \gets \mathcal{N}(0, 1)^{n \times (r + l)}$
        \State $Q, \ \rule{2mm}{0.15mm} \gets qr(A \Omega)$
        \State $U, \ \Sigma, \ V \gets svd(Q^T A)$ \\
        \Return $(Q * U)[\text{:}, 1\text{:}r] * \Sigma[1 \text{:} r, 1 \text{:} r] * V[1 \text{:} r,\text{:}]$
    \end{algorithmic}
    \caption{Randomized SVD.}
    \label{alg:randsvd}
\end{algorithm}


\begin{theorem}
\label{thm:symm}
If $AB^T$ and $B^TA$ are both symmetric matrices, then and only then can two orthogonal matrices $U$ and $V$ be found such that $\Sigma_A = U^T A V$, and $\Sigma_B = U^T B V$ are both diagonal matrices.
\end{theorem}
\begin{proof}
This follows from the spectral theorem ***cite.
\end{proof}

\begin{theorem}
The best (in the Frobenius norm) rank-$r$ approximation to a matrix, $A$, is obtained by the truncated SVD, $A_r$.
\end{theorem}
\begin{proof}
The following proof has been adapted from~\cite{eckart}. The best approximation can be found by
\begin{align}
M = \underset{X \in \bowtie_r}{\argmin} \| A - X \|_F^2,
\end{align}
where $\bowtie_r$ is the set of all rank-$r$ $m \times n$ matrices. The minimum error is then
\begin{align}
\label{eq:err}
\| A - M \|_F^2 &= \left< A, A \right> - 2 \left< A, M \right> + \left< M, M \right>, \\
&= \left< A, A \right> - 2 \left< A, U \Sigma_M V^T \right> \\
\nonumber & \hspace{3cm} + \left< \Sigma_M, \Sigma_M \right>.
\end{align}
At the minimum, the change in $\| A - X \|_F^2$ is zero for some change in $X$. This change in $X$ can be encapsulated as $U \mapsto sU$ where $s$ is infinitesimal and antisymmetric to maintain orthogonality. Thus, at the minimum
\begin{align}
0 = \left< A, s M \right> = \left< A M^T, s \right>.
\end{align}
Therefore, it is the case that $A M^T$ is symmetric. By following a similar procedure, we find that $M^T A$ must be symmetric as well.

By Theorem \ref{thm:symm} $A$ and $M$ exhibit the same $U$ and $V$ in their SVDs. Now \eqref{eq:err} can be simplified to
\begin{align}
\| A - M \|_F^2 &= \| \Sigma_A - \Sigma_M \|_F^2, \\
&= \sum_{i=1}^n \left( \sigma_i(A) - \sigma_i(M) \right)^2, \\
&= \sum_{i = r+1}^n \sigma_i(A)^2.
\end{align}
This minimum is indeed achieved by the truncated SVD, since $\sigma_i(A_r) = \sigma_i(A) H(r - i)$.
\end{proof}

\begin{figure*}
\centering
\begin{subfigure}{0.5\textwidth}
\centering
\begin{gnuplot}[terminal=epslatex, terminaloptions={color size 3.25in,2in lw 3}]
set grid
#set format x '%.1f'
set format y '%.1f'

set ytics 1,0.4

set yr [1:2.2]
set xr [10:240]

set key width -2
set key top left

set xl '$r$'
set yl 'Error'

p 'FullRankBig.dat' u 1:2 w l lc 1 t 'Spectral', '' u 1:($2+$3) w l lc 1 lw 0.3 not, '' u 1:($2 - $3) w l lc 1 lw 0.3 not, \
'' u 1:4 w l t 'Frobenius' lc 2, '' u 1:($4+$5) w l lc 2 lw 0.3 not, '' u 1:($4 - $5) w l lc 2 lw 0.3 not, \
'' u 1:6 w l t 'Trace' lc 7, '' u 1:($6+$7) w l lc 7 lw 0.3 not, '' u 1:($6 - $7) w l lc 7 lw 0.3 not
\end{gnuplot}
\caption{Full rank matrices.}
\label{fig:fullrank}
\end{subfigure}%
\begin{subfigure}{0.5\textwidth}
\centering
\begin{gnuplot}[terminal=epslatex, terminaloptions={color size 3.25in,2in lw 3}]
set grid
#set format x '%.1f'
#set format y '%.1f'

set ytics 1

set yr [1:]
set xr [10:240]

set key width -2
set key top left

set xl '$r$'
set yl 'Error'

p 'PartRankBig.dat' u 1:2 w l lc 1 t 'Spectral', '' u 1:($2+$3/10) w l lc 1 lw 0.3 not, '' u 1:($2 - $3/10) w l lc 1 lw 0.3 not, \
'' u 1:4 w l t 'Frobenius' lc 2, '' u 1:($4+$5/10) w l lc 2 lw 0.3 not, '' u 1:($4 - $5/10) w l lc 2 lw 0.3 not, \
'' u 1:6 w l t 'Trace' lc 7, '' u 1:($6+$7/10) w l lc 7 lw 0.3 not, '' u 1:($6 - $7/10) w l lc 7 lw 0.3 not
\end{gnuplot}
\caption{Rank-$r$ matrices.}
\label{fig:rankr}
\end{subfigure} \\
\begin{subfigure}{0.5\textwidth}
\centering
\begin{gnuplot}[terminal=epslatex, terminaloptions={color size 3.25in,2in lw 3}]
set grid
#set format x '%.1f'
set format y '%.1f'

set ytics 0.5

set yr [1:]
set xr [10:240]

set key vertical maxrows 2
set key width -5
set key bottom

set xl '$r$'
set yl 'Error'

p -1 lc rgb 'white' t ' ', \
'AlgBig.dat' u 1:2 w l lc 1 t 'Spectral', '' u 1:($2+$3) w l lc 1 lw 0.3 not, '' u 1:($2 - $3) w l lc 1 lw 0.3 not, \
'' u 1:4 w l t 'Frobenius' lc 2, '' u 1:($4+$5) w l lc 2 lw 0.3 not, '' u 1:($4 - $5) w l lc 2 lw 0.3 not, \
'' u 1:6 w l t 'Trace' lc 7, '' u 1:($6+$7) w l lc 7 lw 0.3 not, '' u 1:($6 - $7) w l lc 7 lw 0.3 not
\end{gnuplot}
\caption{Algebraically decaying singular values.}
\label{fig:algdecay}
\end{subfigure}%
\begin{subfigure}{0.5\textwidth}
\centering
\begin{gnuplot}[terminal=epslatex, terminaloptions={color size 3.25in,2in lw 3}]
set grid
#set format x '%.1f'
#set format y '%.2f'

set ytics 1, 2

set yr [1:7]
set xr [10:240]

set key width -2
set key top left

set xl '$r$'
set yl 'Error'

p 'GeoBig.dat' u 1:2 w l lc 1 t 'Spectral', '' u 1:($2+$3/10) w l lc 1 lw 0.3 not, '' u 1:($2 - $3/10) w l lc 1 lw 0.3 not, \
'' u 1:4 w l t 'Frobenius' lc 2, '' u 1:($4+$5/10) w l lc 2 lw 0.3 not, '' u 1:($4 - $5/10) w l lc 2 lw 0.3 not, \
'' u 1:6 w l t 'Trace' lc 7, '' u 1:($6+$7/10) w l lc 7 lw 0.3 not, '' u 1:($6 - $7/10) w l lc 7 lw 0.3 not
\end{gnuplot}
\caption{Geometrically decaying singular values.}
\label{fig:geodecay}
\end{subfigure}
\caption{Relative error of four types of matrix for the spectral, Frobenius, and trace norms. The thin lines represent the standard deviation in Figures \ref{fig:fullrank} and \ref{fig:algdecay}, and the standard error in Figures \ref{fig:rankr} and \ref{fig:geodecay} since the calculations are less numerically stable.}
\label{fig:randsvd}
\end{figure*}

\bibliography{Ref}

\end{document}
