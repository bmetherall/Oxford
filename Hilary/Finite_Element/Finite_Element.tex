\documentclass[mathserif]{beamer}
\usepackage{amsmath}
\usepackage{amsfonts}
\usepackage{xcolor,graphicx}
\usepackage{geometry}
\usepackage{hyperref}
\usepackage{mathrsfs}
\usepackage[subfolder,cleanup]{gnuplottex}

\usetheme{InFoMM}

\DeclareMathOperator{\sgn}{sgn}

\title{Finite Element Method for the Tricomi Equation}
\date{March 2020}
\author{Brady Metherall}

\begin{document}
\frame{\titlepage}

\frame{
	\frametitle{Tricomi Equation}
	The Tricomi equation \cite{aziz},
	\begin{align}
		k(y) u_{xx} + u_{yy} = f,
		\label{eq:tricomi}
	\end{align}
	arises when studying transonic flow in fluid mechanics. The function $k$ is assumed to be
	\begin{itemize}
		\item smooth in the domain.
		\item monotonic increasing with $y$.
		\item of the same sign as $y$, that is, $\sgn{k} = \sgn{y}$.
	\end{itemize}
	The most common form is $k(y) = y$, which is analyzed in \cite{trangenstein}.
}

\frame{
	\frametitle{Tricomi Equation}
	We rewrite \eqref{eq:tricomi} as
	\begin{align}
		Lu = f
	\end{align}
	within the domain $\Omega$ bounded by $\Gamma_0$, $\Gamma_1$, and $\Gamma_2$. Here $\Gamma_0$ is a curve in the upper half-plane connecting $(-1,0)$ and $(1,0)$, and $\Gamma_1$ and $\Gamma_2$ are the characteristics passing through $(-1,0)$ and $(1,0)$, respectively. The boundary value problem can now be stated as
	\begin{align}
		L u &= f & \text{in } &\Omega, \\
		u &= 0 & \text{on } &\Gamma_0 \cup \Gamma_1.
	\end{align}
}

\frame{
	\frametitle{Tricomi Domain}
	\begin{figure}
		\centering
		% GNUPLOT: LaTeX picture with Postscript
\begingroup
  \makeatletter
  \providecommand\color[2][]{%
    \GenericError{(gnuplot) \space\space\space\@spaces}{%
      Package color not loaded in conjunction with
      terminal option `colourtext'%
    }{See the gnuplot documentation for explanation.%
    }{Either use 'blacktext' in gnuplot or load the package
      color.sty in LaTeX.}%
    \renewcommand\color[2][]{}%
  }%
  \providecommand\includegraphics[2][]{%
    \GenericError{(gnuplot) \space\space\space\@spaces}{%
      Package graphicx or graphics not loaded%
    }{See the gnuplot documentation for explanation.%
    }{The gnuplot epslatex terminal needs graphicx.sty or graphics.sty.}%
    \renewcommand\includegraphics[2][]{}%
  }%
  \providecommand\rotatebox[2]{#2}%
  \@ifundefined{ifGPcolor}{%
    \newif\ifGPcolor
    \GPcolortrue
  }{}%
  \@ifundefined{ifGPblacktext}{%
    \newif\ifGPblacktext
    \GPblacktextfalse
  }{}%
  % define a \g@addto@macro without @ in the name:
  \let\gplgaddtomacro\g@addto@macro
  % define empty templates for all commands taking text:
  \gdef\gplbacktext{}%
  \gdef\gplfronttext{}%
  \makeatother
  \ifGPblacktext
    % no textcolor at all
    \def\colorrgb#1{}%
    \def\colorgray#1{}%
  \else
    % gray or color?
    \ifGPcolor
      \def\colorrgb#1{\color[rgb]{#1}}%
      \def\colorgray#1{\color[gray]{#1}}%
      \expandafter\def\csname LTw\endcsname{\color{white}}%
      \expandafter\def\csname LTb\endcsname{\color{black}}%
      \expandafter\def\csname LTa\endcsname{\color{black}}%
      \expandafter\def\csname LT0\endcsname{\color[rgb]{1,0,0}}%
      \expandafter\def\csname LT1\endcsname{\color[rgb]{0,1,0}}%
      \expandafter\def\csname LT2\endcsname{\color[rgb]{0,0,1}}%
      \expandafter\def\csname LT3\endcsname{\color[rgb]{1,0,1}}%
      \expandafter\def\csname LT4\endcsname{\color[rgb]{0,1,1}}%
      \expandafter\def\csname LT5\endcsname{\color[rgb]{1,1,0}}%
      \expandafter\def\csname LT6\endcsname{\color[rgb]{0,0,0}}%
      \expandafter\def\csname LT7\endcsname{\color[rgb]{1,0.3,0}}%
      \expandafter\def\csname LT8\endcsname{\color[rgb]{0.5,0.5,0.5}}%
    \else
      % gray
      \def\colorrgb#1{\color{black}}%
      \def\colorgray#1{\color[gray]{#1}}%
      \expandafter\def\csname LTw\endcsname{\color{white}}%
      \expandafter\def\csname LTb\endcsname{\color{black}}%
      \expandafter\def\csname LTa\endcsname{\color{black}}%
      \expandafter\def\csname LT0\endcsname{\color{black}}%
      \expandafter\def\csname LT1\endcsname{\color{black}}%
      \expandafter\def\csname LT2\endcsname{\color{black}}%
      \expandafter\def\csname LT3\endcsname{\color{black}}%
      \expandafter\def\csname LT4\endcsname{\color{black}}%
      \expandafter\def\csname LT5\endcsname{\color{black}}%
      \expandafter\def\csname LT6\endcsname{\color{black}}%
      \expandafter\def\csname LT7\endcsname{\color{black}}%
      \expandafter\def\csname LT8\endcsname{\color{black}}%
    \fi
  \fi
    \setlength{\unitlength}{0.0500bp}%
    \ifx\gptboxheight\undefined%
      \newlength{\gptboxheight}%
      \newlength{\gptboxwidth}%
      \newsavebox{\gptboxtext}%
    \fi%
    \setlength{\fboxrule}{0.5pt}%
    \setlength{\fboxsep}{1pt}%
\begin{picture}(5760.00,3600.00)%
    \gplgaddtomacro\gplbacktext{%
    }%
    \gplgaddtomacro\gplfronttext{%
      \csname LTb\endcsname%%
      \put(198,2041){\makebox(0,0){\strut{}$y$}}%
      \put(3176,154){\makebox(0,0){\strut{}$x$}}%
      \csname LTb\endcsname%%
      \put(858,704){\makebox(0,0)[r]{\strut{}$-1.5$}}%
      \csname LTb\endcsname%%
      \put(858,1190){\makebox(0,0)[r]{\strut{}$-1.0$}}%
      \csname LTb\endcsname%%
      \put(858,1677){\makebox(0,0)[r]{\strut{}$-0.5$}}%
      \csname LTb\endcsname%%
      \put(858,2163){\makebox(0,0)[r]{\strut{}$0.0$}}%
      \csname LTb\endcsname%%
      \put(858,2649){\makebox(0,0)[r]{\strut{}$0.5$}}%
      \csname LTb\endcsname%%
      \put(858,3136){\makebox(0,0)[r]{\strut{}$1.0$}}%
      \csname LTb\endcsname%%
      \put(990,484){\makebox(0,0){\strut{}$-1.0$}}%
      \csname LTb\endcsname%%
      \put(2083,484){\makebox(0,0){\strut{}$-0.5$}}%
      \csname LTb\endcsname%%
      \put(3177,484){\makebox(0,0){\strut{}$0.0$}}%
      \csname LTb\endcsname%%
      \put(4270,484){\makebox(0,0){\strut{}$0.5$}}%
      \csname LTb\endcsname%%
      \put(5363,484){\makebox(0,0){\strut{}$1.0$}}%
      \colorrgb{0.58,0.00,0.83}%%
      \put(3177,2163){\makebox(0,0){\strut{}$\Omega$}}%
      \colorrgb{0.00,0.62,0.45}%%
      \put(4816,2893){\makebox(0,0){\strut{}$\Gamma_0$}}%
      \colorrgb{0.90,0.62,0.00}%%
      \put(1537,1434){\makebox(0,0){\strut{}$\Gamma_1$}}%
      \colorrgb{0.34,0.71,0.91}%%
      \put(4816,1434){\makebox(0,0){\strut{}$\Gamma_2$}}%
    }%
    \gplbacktext
    \put(0,0){\includegraphics{Domain}}%
    \gplfronttext
  \end{picture}%
\endgroup

		\caption{Example domain of $y u_{xx} + u_{yy} = 0$.}
		\label{fig:dom}
	\end{figure}
}

\frame{
	\frametitle{Weak Formulation}
	We define the function space, $W$, as
	\begin{align*}
		W = \{ \phi \in C(\bar{\Omega}) \cap H^2(\Omega) \cap H^1(\partial \Omega) &: \\
		L \phi \in L^2(\Omega) &\text{ and } \phi|_{\Gamma_0 \cup \Gamma_1} = 0 \}.
	\end{align*}
	Then the weak formulation can be phrased as: given $f \in L^2(\Omega)$, we wish to find $u \in W$ such that
	\begin{align}
		(Lu, lv)_{L^2(\Omega)} = (f, lv)_{L^2(\Omega)}
	\end{align}
	for all $v \in W$. $l : W \rightarrow L^2(\Omega)$ is defined as
	\begin{align}
		l \phi = \alpha_1 \phi_x + \alpha_2 \phi_y
	\end{align}
	for all $\phi \in W$, where $\alpha_1$ and $\alpha_2$ are known functions.
}

% \frame{
% 	\frametitle{Continuity and Coercivity}
% 	Continuity follows from Cauchy-Schwarz
% }

\frame{
	\frametitle{Finite Elements}
	We let $\{ V_h \}$ be a family of finite dimensional subspaces $W$, and define an approximate solution, $u_h \in V_h$, to be one where given $f \in L^2(\Omega)$
	\begin{align}
		(Lu_h, lv_h)_{L^2(\Omega)} = (f, lv_h)_{L^2(\Omega)}
		\label{eq:finite}
	\end{align}
	for all $v_h \in V_h$.

	Let $\{ \phi_j \}$ be a basis for $V_h$, then \eqref{eq:finite} can be cast into the linear system
	\begin{align}
		\mathbf{Au} = \mathbf{b},
	\end{align}
	where $A_{ij} = (L \phi_j, l \phi_i)_{L^2(\Omega)}$, $b_i = (f, l \phi_i)_{L^2(\Omega)}$, and $u_h = \sum_i u_i \phi_i$.
}

\frame{
	\frametitle{Error Analysis}

}

\frame{
	\frametitle{References}
	\bibliography{Ref}
}


\end{document}
