\documentclass[mathserif]{beamer}
\usepackage{amsmath}
\usepackage{amsfonts}
\usepackage{xcolor,graphicx}
\usepackage{geometry}
\usepackage{hyperref}
\usepackage{mathrsfs}
\usepackage[subfolder,cleanup]{gnuplottex}

\usetheme{InFoMM}

\DeclareMathOperator{\sgn}{sgn}

% \definecolor{mycolour}{RGB}{150, 150, 255}
% \setbeamercolor{block body}{bg=mycolour,fg=black}

\title{Finite Element Method for the Tricomi Equation}
\date{March 2020}
\author{Brady Metherall}

\begin{document}
\frame{\titlepage}

\frame{
	\frametitle{Tricomi Equation}
	The Tricomi equation \cite{aziz},
	\begin{align}
		k(y) u_{xx} + u_{yy} = f,
		\label{eq:tricomi}
	\end{align}
	arises when studying transonic flow in fluid mechanics. The function $k$ is assumed to be
	\begin{itemize}
		\item smooth in the domain.
		\item monotonic increasing with $y$.
		\item of the same sign as $y$, that is, $\sgn{k} = \sgn{y}$.
	\end{itemize}
	The most common form is $k(y) = y$, which is analyzed in \cite{trangenstein}.
}

\frame{
	\frametitle{Tricomi Equation}
	We rewrite \eqref{eq:tricomi} as
	\begin{align}
		Lu = f
	\end{align}
	within the domain $\Omega$ bounded by $\Gamma_0$, $\Gamma_1$, and $\Gamma_2$. Here $\Gamma_0$ is a curve in the upper half-plane connecting $(-1,0)$ and $(1,0)$, and $\Gamma_1$ and $\Gamma_2$ are the characteristics passing through $(-1,0)$ and $(1,0)$, respectively. The boundary value problem can now be stated as
	\begin{align}
		L u &= f & \text{in } &\Omega, \\
		u &= 0 & \text{on } &\Gamma_0 \cup \Gamma_1.
	\end{align}
}

\frame{
	\frametitle{Tricomi Domain}
	\begin{figure}
		\centering
		\input{Domain}
		\caption{Example domain of $y u_{xx} + u_{yy} = 0$.}
		\label{fig:dom}
	\end{figure}
}

\frame{
	\frametitle{Weak Formulation}
	We define the function space, $W$, as
	\begin{align*}
		W = \{ \phi \in C(\bar{\Omega}) \cap H^2(\Omega) \cap H^1(\partial \Omega) &: \\
		L \phi \in L^2(\Omega) &\text{ and } \phi|_{\Gamma_0 \cup \Gamma_1} = 0 \}.
	\end{align*}
	Then the weak formulation can be phrased as: given $f \in L^2(\Omega)$, we wish to find $u \in W$ such that
	\begin{align}
		(Lu, lv)_{L^2(\Omega)} = (f, lv)_{L^2(\Omega)}
	\end{align}
	for all $v \in W$. $l : W \rightarrow L^2(\Omega)$ is defined as
	\begin{align}
		l \phi = \alpha_1 \phi_x + \alpha_2 \phi_y
	\end{align}
	for all $\phi \in W$, where $\alpha_1$ and $\alpha_2$ are known functions.
}

% \frame{
% 	\frametitle{Continuity and Coercivity}
% 	Continuity follows from Cauchy-Schwarz
% }

\frame{
	\frametitle{Finite Elements}
	We let $\{ V_h \}$ be a family of finite dimensional subspaces $W$, and define an approximate solution, $u_h \in V_h$, to be one where given $f \in L^2(\Omega)$
	\begin{align}
		(Lu_h, lv_h)_{L^2(\Omega)} = (f, lv_h)_{L^2(\Omega)}
		\label{eq:finite}
	\end{align}
	for all $v_h \in V_h$.

	Let $\{ \phi_j \}$ be a basis for $V_h$, then \eqref{eq:finite} can be cast into the linear system
	\begin{align}
		\mathbf{Au} = \mathbf{b},
	\end{align}
	where $A_{ij} = (L \phi_j, l \phi_i)_{L^2(\Omega)}$, $b_i = (f, l \phi_i)_{L^2(\Omega)}$, and $u_h = \sum_i u_i \phi_i$.
}

\frame{
	\frametitle{Finite Elements}
	\begin{lemma}
		The matrix $\mathbf{A}$ is invertible.
	\end{lemma}
	\begin{proof}
		Suppose there exists a vector $\mathbf{z}$ such that $\mathbf{Az} = \mathbf{0}$, and let $z_h = \sum_i z_i \phi_i$. Then
		\begin{align}
			0 &= \mathbf{z}^T \mathbf{Az}, \\
			&= (L z_h, l z_h), \\
			&\geq C \| z_h \|^2
		\end{align}
		by coercivity. Thus, $z_h = 0$ and $\mathbf{z} = \mathbf{0}$, and so $\mathbf{A}$ is invertible.
	\end{proof}
}

\frame{
	\frametitle{Finite Elements}
	\begin{theorem}
		There is a unique $u_h \in V_h$ satisfying the weak formulation
		\begin{align}
			(Lu_h, lv_h)_{L^2(\Omega)} = (f, lv_h)_{L^2(\Omega)}.
		\end{align}
	\end{theorem}
	\begin{proof}
		This follows immediately from the previous lemma since $\mathbf{A}$ has a non-zero determinant.
	\end{proof}
}

\frame{
	\frametitle{Error Analysis}

}

\frame{
	\frametitle{References}
	\bibliography{Ref}
}


\end{document}
