\documentclass[11pt,a4paper,twocolumn]{article}

\usepackage{amsfonts}
\usepackage{amsmath}
\usepackage{amssymb}
\usepackage{geometry}
\usepackage{xcolor}
\usepackage{graphicx}
%\usepackage[subfolder,cleanup]{gnuplottex}
%\usepackage{amsthm}
%\usepackage{enumitem}
%\usepackage{wrapfig}
\usepackage{subcaption}
%\usepackage{hyperref}
%\usepackage{tikz}
\usepackage{makecell}
%\usepackage{dblfloatfix}

\usepackage{lipsum}

% Nicer brackets for operators
\let\originalleft\left
\let\originalright\right
\renewcommand{\left}{\mathopen{}\mathclose\bgroup\originalleft}
\renewcommand{\right}{\aftergroup\egroup\originalright}

% Math operators
\providecommand{\bigO}[1]{\ensuremath{\mathop{}\mathopen{}\mathcal{O}\mathopen{}\left(#1\right)}}

% Macros
\newcommand{\diff}[3][\hspace{-0.5pt}]{\frac{\textrm{d}^{#1}#2}{\textrm{d}{#3}^{#1}}}
\newcommand{\pdiff}[3][\hspace{-0.5pt}]{\frac{\partial^{#1}#2}{\partial{#3}^{#1}}}
\newcommand{\df}{\, \textrm{d}}
%\newcommand{\eps}{\varepsilon}

% Row colouring in tables
%\usepackage[table]{xcolor}
%\rowcolors{2}{gray!25}{white}

% Margin size
\newgeometry{margin=2cm}

%\renewcommand\bottomfraction{.75}

% Reference style
%\bibliographystyle{ieeetr}

\title{Continuous Optimization \\ InFoMM Assignment}
\author{Brady Metherall}
\date{16 March, 2020}

\begin{document}
\maketitle

\section{Rosenbrock Function}

unconstrained rosenbrock

quadratic constrained rosenbrock wikipedia


\begin{align}
	\min_{x, y \in \mathbb{R}} 100(y - x^2)^2 + (1 - x)^2
\end{align}

\begin{align}
	(x - 1)^2 + y &\geq 1 \\
	x + y &\leq 2
\end{align}

\begin{figure}[tbp]
	\centering
	% GNUPLOT: LaTeX picture with Postscript
\begingroup
  \makeatletter
  \providecommand\color[2][]{%
    \GenericError{(gnuplot) \space\space\space\@spaces}{%
      Package color not loaded in conjunction with
      terminal option `colourtext'%
    }{See the gnuplot documentation for explanation.%
    }{Either use 'blacktext' in gnuplot or load the package
      color.sty in LaTeX.}%
    \renewcommand\color[2][]{}%
  }%
  \providecommand\includegraphics[2][]{%
    \GenericError{(gnuplot) \space\space\space\@spaces}{%
      Package graphicx or graphics not loaded%
    }{See the gnuplot documentation for explanation.%
    }{The gnuplot epslatex terminal needs graphicx.sty or graphics.sty.}%
    \renewcommand\includegraphics[2][]{}%
  }%
  \providecommand\rotatebox[2]{#2}%
  \@ifundefined{ifGPcolor}{%
    \newif\ifGPcolor
    \GPcolortrue
  }{}%
  \@ifundefined{ifGPblacktext}{%
    \newif\ifGPblacktext
    \GPblacktexttrue
  }{}%
  % define a \g@addto@macro without @ in the name:
  \let\gplgaddtomacro\g@addto@macro
  % define empty templates for all commands taking text:
  \gdef\gplbacktext{}%
  \gdef\gplfronttext{}%
  \makeatother
  \ifGPblacktext
    % no textcolor at all
    \def\colorrgb#1{}%
    \def\colorgray#1{}%
  \else
    % gray or color?
    \ifGPcolor
      \def\colorrgb#1{\color[rgb]{#1}}%
      \def\colorgray#1{\color[gray]{#1}}%
      \expandafter\def\csname LTw\endcsname{\color{white}}%
      \expandafter\def\csname LTb\endcsname{\color{black}}%
      \expandafter\def\csname LTa\endcsname{\color{black}}%
      \expandafter\def\csname LT0\endcsname{\color[rgb]{1,0,0}}%
      \expandafter\def\csname LT1\endcsname{\color[rgb]{0,1,0}}%
      \expandafter\def\csname LT2\endcsname{\color[rgb]{0,0,1}}%
      \expandafter\def\csname LT3\endcsname{\color[rgb]{1,0,1}}%
      \expandafter\def\csname LT4\endcsname{\color[rgb]{0,1,1}}%
      \expandafter\def\csname LT5\endcsname{\color[rgb]{1,1,0}}%
      \expandafter\def\csname LT6\endcsname{\color[rgb]{0,0,0}}%
      \expandafter\def\csname LT7\endcsname{\color[rgb]{1,0.3,0}}%
      \expandafter\def\csname LT8\endcsname{\color[rgb]{0.5,0.5,0.5}}%
    \else
      % gray
      \def\colorrgb#1{\color{black}}%
      \def\colorgray#1{\color[gray]{#1}}%
      \expandafter\def\csname LTw\endcsname{\color{white}}%
      \expandafter\def\csname LTb\endcsname{\color{black}}%
      \expandafter\def\csname LTa\endcsname{\color{black}}%
      \expandafter\def\csname LT0\endcsname{\color{black}}%
      \expandafter\def\csname LT1\endcsname{\color{black}}%
      \expandafter\def\csname LT2\endcsname{\color{black}}%
      \expandafter\def\csname LT3\endcsname{\color{black}}%
      \expandafter\def\csname LT4\endcsname{\color{black}}%
      \expandafter\def\csname LT5\endcsname{\color{black}}%
      \expandafter\def\csname LT6\endcsname{\color{black}}%
      \expandafter\def\csname LT7\endcsname{\color{black}}%
      \expandafter\def\csname LT8\endcsname{\color{black}}%
    \fi
  \fi
    \setlength{\unitlength}{0.0500bp}%
    \ifx\gptboxheight\undefined%
      \newlength{\gptboxheight}%
      \newlength{\gptboxwidth}%
      \newsavebox{\gptboxtext}%
    \fi%
    \setlength{\fboxrule}{0.5pt}%
    \setlength{\fboxsep}{1pt}%
\begin{picture}(4536.00,2880.00)%
    \gplgaddtomacro\gplbacktext{%
      \csname LTb\endcsname%%
      \put(2939,108){\makebox(0,0){\strut{}$-1$}}%
      \csname LTb\endcsname%%
      \put(3605,401){\makebox(0,0){\strut{}$0$}}%
      \csname LTb\endcsname%%
      \put(4270,694){\makebox(0,0){\strut{}$1$}}%
      \csname LTb\endcsname%%
      \put(1992,152){\makebox(0,0){\strut{}$-1$}}%
      \csname LTb\endcsname%%
      \put(1433,501){\makebox(0,0){\strut{}$0$}}%
      \csname LTb\endcsname%%
      \put(875,851){\makebox(0,0){\strut{}$1$}}%
      \put(647,1107){\makebox(0,0)[r]{\strut{}$0$}}%
      \put(647,1515){\makebox(0,0)[r]{\strut{}$400$}}%
      \put(647,1922){\makebox(0,0)[r]{\strut{}$800$}}%
      \put(647,2330){\makebox(0,0)[r]{\strut{}$1200$}}%
      \put(647,2737){\makebox(0,0)[r]{\strut{}$1600$}}%
    }%
    \gplgaddtomacro\gplfronttext{%
      \csname LTb\endcsname%%
      \put(3893,218){\makebox(0,0){\strut{}$x$}}%
      \put(1076,348){\makebox(0,0){\strut{}$y$}}%
      \put(76,1951){\makebox(0,0){\strut{}$z$}}%
    }%
    \gplbacktext
    \put(0,0){\includegraphics{Rosen}}%
    \gplfronttext
  \end{picture}%
\endgroup

	\caption{}
	\label{fig:rosen}
\end{figure}

\begin{figure}[tbp]
	\centering
	% GNUPLOT: LaTeX picture with Postscript
\begingroup
  \makeatletter
  \providecommand\color[2][]{%
    \GenericError{(gnuplot) \space\space\space\@spaces}{%
      Package color not loaded in conjunction with
      terminal option `colourtext'%
    }{See the gnuplot documentation for explanation.%
    }{Either use 'blacktext' in gnuplot or load the package
      color.sty in LaTeX.}%
    \renewcommand\color[2][]{}%
  }%
  \providecommand\includegraphics[2][]{%
    \GenericError{(gnuplot) \space\space\space\@spaces}{%
      Package graphicx or graphics not loaded%
    }{See the gnuplot documentation for explanation.%
    }{The gnuplot epslatex terminal needs graphicx.sty or graphics.sty.}%
    \renewcommand\includegraphics[2][]{}%
  }%
  \providecommand\rotatebox[2]{#2}%
  \@ifundefined{ifGPcolor}{%
    \newif\ifGPcolor
    \GPcolortrue
  }{}%
  \@ifundefined{ifGPblacktext}{%
    \newif\ifGPblacktext
    \GPblacktexttrue
  }{}%
  % define a \g@addto@macro without @ in the name:
  \let\gplgaddtomacro\g@addto@macro
  % define empty templates for all commands taking text:
  \gdef\gplbacktext{}%
  \gdef\gplfronttext{}%
  \makeatother
  \ifGPblacktext
    % no textcolor at all
    \def\colorrgb#1{}%
    \def\colorgray#1{}%
  \else
    % gray or color?
    \ifGPcolor
      \def\colorrgb#1{\color[rgb]{#1}}%
      \def\colorgray#1{\color[gray]{#1}}%
      \expandafter\def\csname LTw\endcsname{\color{white}}%
      \expandafter\def\csname LTb\endcsname{\color{black}}%
      \expandafter\def\csname LTa\endcsname{\color{black}}%
      \expandafter\def\csname LT0\endcsname{\color[rgb]{1,0,0}}%
      \expandafter\def\csname LT1\endcsname{\color[rgb]{0,1,0}}%
      \expandafter\def\csname LT2\endcsname{\color[rgb]{0,0,1}}%
      \expandafter\def\csname LT3\endcsname{\color[rgb]{1,0,1}}%
      \expandafter\def\csname LT4\endcsname{\color[rgb]{0,1,1}}%
      \expandafter\def\csname LT5\endcsname{\color[rgb]{1,1,0}}%
      \expandafter\def\csname LT6\endcsname{\color[rgb]{0,0,0}}%
      \expandafter\def\csname LT7\endcsname{\color[rgb]{1,0.3,0}}%
      \expandafter\def\csname LT8\endcsname{\color[rgb]{0.5,0.5,0.5}}%
    \else
      % gray
      \def\colorrgb#1{\color{black}}%
      \def\colorgray#1{\color[gray]{#1}}%
      \expandafter\def\csname LTw\endcsname{\color{white}}%
      \expandafter\def\csname LTb\endcsname{\color{black}}%
      \expandafter\def\csname LTa\endcsname{\color{black}}%
      \expandafter\def\csname LT0\endcsname{\color{black}}%
      \expandafter\def\csname LT1\endcsname{\color{black}}%
      \expandafter\def\csname LT2\endcsname{\color{black}}%
      \expandafter\def\csname LT3\endcsname{\color{black}}%
      \expandafter\def\csname LT4\endcsname{\color{black}}%
      \expandafter\def\csname LT5\endcsname{\color{black}}%
      \expandafter\def\csname LT6\endcsname{\color{black}}%
      \expandafter\def\csname LT7\endcsname{\color{black}}%
      \expandafter\def\csname LT8\endcsname{\color{black}}%
    \fi
  \fi
    \setlength{\unitlength}{0.0500bp}%
    \ifx\gptboxheight\undefined%
      \newlength{\gptboxheight}%
      \newlength{\gptboxwidth}%
      \newsavebox{\gptboxtext}%
    \fi%
    \setlength{\fboxrule}{0.5pt}%
    \setlength{\fboxsep}{1pt}%
\begin{picture}(4536.00,2880.00)%
    \gplgaddtomacro\gplbacktext{%
      \csname LTb\endcsname%%
      \put(940,851){\makebox(0,0){\strut{}$-1$}}%
      \csname LTb\endcsname%%
      \put(1498,501){\makebox(0,0){\strut{}$0$}}%
      \csname LTb\endcsname%%
      \put(2057,152){\makebox(0,0){\strut{}$1$}}%
      \csname LTb\endcsname%%
      \put(3005,108){\makebox(0,0){\strut{}$-1$}}%
      \csname LTb\endcsname%%
      \put(3671,401){\makebox(0,0){\strut{}$0$}}%
      \csname LTb\endcsname%%
      \put(4336,694){\makebox(0,0){\strut{}$1$}}%
      \put(647,1107){\makebox(0,0)[r]{\strut{}$0$}}%
      \put(647,1515){\makebox(0,0)[r]{\strut{}$400$}}%
      \put(647,1922){\makebox(0,0)[r]{\strut{}$800$}}%
      \put(647,2330){\makebox(0,0)[r]{\strut{}$1200$}}%
      \put(647,2737){\makebox(0,0)[r]{\strut{}$1600$}}%
    }%
    \gplgaddtomacro\gplfronttext{%
      \csname LTb\endcsname%%
      \put(1076,348){\makebox(0,0){\strut{}$x$}}%
      \put(3893,218){\makebox(0,0){\strut{}$y$}}%
      \put(76,1951){\makebox(0,0){\strut{}$z$}}%
    }%
    \gplbacktext
    \put(0,0){\includegraphics{RosenConst}}%
    \gplfronttext
  \end{picture}%
\endgroup

	\caption{}
	\label{fig:rosenconst}
\end{figure}

\begin{table*}[tbp]
	\centering
	\begin{tabular}{lcccccc}
		Starting value & $(-2,-3)$ & $(2,1)$ & $(-3,4)$ & $(\pi,-\textrm{e})$ & $(0,0)$ & $(2,4)$ \\
		\noalign{\smallskip}\Xhline{2.5\arrayrulewidth}\hline\noalign{\smallskip}
		\multicolumn{7}{c}{Unconstrained} \\
		\noalign{\smallskip}\hline\noalign{\smallskip}
		No derivatives & 15 & 43 & 28 & 35 & 22 & 35 \\
		Gradient & 15 & 41 & 28 & 35 & DNC\footnotemark & DNC \\
		Hessian & 25 & 13 & 28 & 20 & DNC & DNC \\
		\noalign{\smallskip}\hline\noalign{\smallskip}
		\multicolumn{7}{c}{Constrained} \\
		\noalign{\smallskip}\hline\noalign{\smallskip}
		No derivatives & 36 & 16 & 40 & 22 & 17 & 18 \\
		Gradient & 36 & 16 & 41 & 22 & DNC & DNC \\
		Hessian & 20 & 12 & 23 & 12 & DNC & DNC \\
		\noalign{\smallskip}\hline\noalign{\smallskip}
	\end{tabular} \\
	\footnotesize{$^1$Did not converge.}
	\caption{Iterations needed to converge to the minimum of the constrained and unconstrained Rosenbrock function.}
	\label{tab:iterations}
\end{table*}



\section{Catenary Problem}
We now turn our attention to the catenary problem. The catenary problem is concerned with the shape of a freely hanging rope with fixed endpoints neglecting bending stiffness. The solution is the shape that minimizes the gravitational potential energy, and thus is a staple in calculus of variations and mechanics courses. In this section we focus our attention on the discrete analogue of this problem. We instead consider a series of $n$ rigid beams attached together. By assuming the gravitational potential energy is focused entirely at the centre of each beam, we obtain the expression
\begin{align}
	U = m g \left( \frac{1}{2} y_0 + \frac{1}{2} y_n + \sum_{i = 1}^{n-1} y_i \right)
\end{align}
for the total potential energy, as the midpoint is the average of the endpoints of each beam. Additionally, there are multiple constraints of the system. We assume both endpoints are fixed at the same height, and that each beam is the same length, $L$. Combining these constraints with the expression for the potential energy yields the optimization problem
\begin{gather}
	\min_{\mathbf{x}, \mathbf{y} \in \mathbb{R}^{n+1}} m g \left( \frac{1}{2} y_0 + \frac{1}{2} y_n + \sum_{i = 1}^{n-1} y_i \right) \\
	\begin{align}
		x_0 &= 0, & x_n &= \gamma n L, \nonumber \\
		y_0 &=0, & y_n &= 0,
	\end{align} \\
	(x_i - x_{i+1})^2 + (y_i - y_{i+1})^2 = L^2, \nonumber
\end{gather}
where $\gamma$ is the ratio of the width of the span to the total length of the beams. While solving the problem we use the the following numerical values:
\begin{gather}
	\begin{align}
		x_n &= 10, & g &= 1, & m &= 1,
	\end{align} \\
	\begin{align}
		\gamma &= \frac{4}{5}, & L &= \frac{x_n}{\gamma n}.
	\end{align}
\end{gather}

\begin{figure*}[tbp]
	\centering
	\begin{subfigure}{\textwidth}
		\centering
		% GNUPLOT: LaTeX picture with Postscript
\begingroup
  \makeatletter
  \providecommand\color[2][]{%
    \GenericError{(gnuplot) \space\space\space\@spaces}{%
      Package color not loaded in conjunction with
      terminal option `colourtext'%
    }{See the gnuplot documentation for explanation.%
    }{Either use 'blacktext' in gnuplot or load the package
      color.sty in LaTeX.}%
    \renewcommand\color[2][]{}%
  }%
  \providecommand\includegraphics[2][]{%
    \GenericError{(gnuplot) \space\space\space\@spaces}{%
      Package graphicx or graphics not loaded%
    }{See the gnuplot documentation for explanation.%
    }{The gnuplot epslatex terminal needs graphicx.sty or graphics.sty.}%
    \renewcommand\includegraphics[2][]{}%
  }%
  \providecommand\rotatebox[2]{#2}%
  \@ifundefined{ifGPcolor}{%
    \newif\ifGPcolor
    \GPcolortrue
  }{}%
  \@ifundefined{ifGPblacktext}{%
    \newif\ifGPblacktext
    \GPblacktexttrue
  }{}%
  % define a \g@addto@macro without @ in the name:
  \let\gplgaddtomacro\g@addto@macro
  % define empty templates for all commands taking text:
  \gdef\gplbacktext{}%
  \gdef\gplfronttext{}%
  \makeatother
  \ifGPblacktext
    % no textcolor at all
    \def\colorrgb#1{}%
    \def\colorgray#1{}%
  \else
    % gray or color?
    \ifGPcolor
      \def\colorrgb#1{\color[rgb]{#1}}%
      \def\colorgray#1{\color[gray]{#1}}%
      \expandafter\def\csname LTw\endcsname{\color{white}}%
      \expandafter\def\csname LTb\endcsname{\color{black}}%
      \expandafter\def\csname LTa\endcsname{\color{black}}%
      \expandafter\def\csname LT0\endcsname{\color[rgb]{1,0,0}}%
      \expandafter\def\csname LT1\endcsname{\color[rgb]{0,1,0}}%
      \expandafter\def\csname LT2\endcsname{\color[rgb]{0,0,1}}%
      \expandafter\def\csname LT3\endcsname{\color[rgb]{1,0,1}}%
      \expandafter\def\csname LT4\endcsname{\color[rgb]{0,1,1}}%
      \expandafter\def\csname LT5\endcsname{\color[rgb]{1,1,0}}%
      \expandafter\def\csname LT6\endcsname{\color[rgb]{0,0,0}}%
      \expandafter\def\csname LT7\endcsname{\color[rgb]{1,0.3,0}}%
      \expandafter\def\csname LT8\endcsname{\color[rgb]{0.5,0.5,0.5}}%
    \else
      % gray
      \def\colorrgb#1{\color{black}}%
      \def\colorgray#1{\color[gray]{#1}}%
      \expandafter\def\csname LTw\endcsname{\color{white}}%
      \expandafter\def\csname LTb\endcsname{\color{black}}%
      \expandafter\def\csname LTa\endcsname{\color{black}}%
      \expandafter\def\csname LT0\endcsname{\color{black}}%
      \expandafter\def\csname LT1\endcsname{\color{black}}%
      \expandafter\def\csname LT2\endcsname{\color{black}}%
      \expandafter\def\csname LT3\endcsname{\color{black}}%
      \expandafter\def\csname LT4\endcsname{\color{black}}%
      \expandafter\def\csname LT5\endcsname{\color{black}}%
      \expandafter\def\csname LT6\endcsname{\color{black}}%
      \expandafter\def\csname LT7\endcsname{\color{black}}%
      \expandafter\def\csname LT8\endcsname{\color{black}}%
    \fi
  \fi
    \setlength{\unitlength}{0.0500bp}%
    \ifx\gptboxheight\undefined%
      \newlength{\gptboxheight}%
      \newlength{\gptboxwidth}%
      \newsavebox{\gptboxtext}%
    \fi%
    \setlength{\fboxrule}{0.5pt}%
    \setlength{\fboxsep}{1pt}%
\begin{picture}(7200.00,3240.00)%
    \gplgaddtomacro\gplbacktext{%
      \csname LTb\endcsname%%
      \put(739,704){\makebox(0,0)[r]{\strut{}$-4$}}%
      \csname LTb\endcsname%%
      \put(739,1283){\makebox(0,0)[r]{\strut{}$-3$}}%
      \csname LTb\endcsname%%
      \put(739,1862){\makebox(0,0)[r]{\strut{}$-2$}}%
      \csname LTb\endcsname%%
      \put(739,2440){\makebox(0,0)[r]{\strut{}$-1$}}%
      \csname LTb\endcsname%%
      \put(739,3019){\makebox(0,0)[r]{\strut{}$0$}}%
      \csname LTb\endcsname%%
      \put(871,484){\makebox(0,0){\strut{}$0$}}%
      \csname LTb\endcsname%%
      \put(2028,484){\makebox(0,0){\strut{}$2$}}%
      \csname LTb\endcsname%%
      \put(3186,484){\makebox(0,0){\strut{}$4$}}%
      \csname LTb\endcsname%%
      \put(4343,484){\makebox(0,0){\strut{}$6$}}%
      \csname LTb\endcsname%%
      \put(5501,484){\makebox(0,0){\strut{}$8$}}%
      \csname LTb\endcsname%%
      \put(6658,484){\makebox(0,0){\strut{}$10$}}%
    }%
    \gplgaddtomacro\gplfronttext{%
      \csname LTb\endcsname%%
      \put(343,1861){\makebox(0,0){\strut{}$y$}}%
      \put(3764,154){\makebox(0,0){\strut{}$x$}}%
      \csname LTb\endcsname%%
      \put(3997,2846){\makebox(0,0)[r]{\strut{}Four Beams}}%
      \csname LTb\endcsname%%
      \put(3997,2626){\makebox(0,0)[r]{\strut{}Continuous}}%
    }%
    \gplbacktext
    \put(0,0){\includegraphics{Beam5}}%
    \gplfronttext
  \end{picture}%
\endgroup

		\caption{$n = 4$.}
		\label{fig:beam4}
	\end{subfigure} \\
	\begin{subfigure}{\textwidth}
		\centering
		% GNUPLOT: LaTeX picture with Postscript
\begingroup
  \makeatletter
  \providecommand\color[2][]{%
    \GenericError{(gnuplot) \space\space\space\@spaces}{%
      Package color not loaded in conjunction with
      terminal option `colourtext'%
    }{See the gnuplot documentation for explanation.%
    }{Either use 'blacktext' in gnuplot or load the package
      color.sty in LaTeX.}%
    \renewcommand\color[2][]{}%
  }%
  \providecommand\includegraphics[2][]{%
    \GenericError{(gnuplot) \space\space\space\@spaces}{%
      Package graphicx or graphics not loaded%
    }{See the gnuplot documentation for explanation.%
    }{The gnuplot epslatex terminal needs graphicx.sty or graphics.sty.}%
    \renewcommand\includegraphics[2][]{}%
  }%
  \providecommand\rotatebox[2]{#2}%
  \@ifundefined{ifGPcolor}{%
    \newif\ifGPcolor
    \GPcolortrue
  }{}%
  \@ifundefined{ifGPblacktext}{%
    \newif\ifGPblacktext
    \GPblacktexttrue
  }{}%
  % define a \g@addto@macro without @ in the name:
  \let\gplgaddtomacro\g@addto@macro
  % define empty templates for all commands taking text:
  \gdef\gplbacktext{}%
  \gdef\gplfronttext{}%
  \makeatother
  \ifGPblacktext
    % no textcolor at all
    \def\colorrgb#1{}%
    \def\colorgray#1{}%
  \else
    % gray or color?
    \ifGPcolor
      \def\colorrgb#1{\color[rgb]{#1}}%
      \def\colorgray#1{\color[gray]{#1}}%
      \expandafter\def\csname LTw\endcsname{\color{white}}%
      \expandafter\def\csname LTb\endcsname{\color{black}}%
      \expandafter\def\csname LTa\endcsname{\color{black}}%
      \expandafter\def\csname LT0\endcsname{\color[rgb]{1,0,0}}%
      \expandafter\def\csname LT1\endcsname{\color[rgb]{0,1,0}}%
      \expandafter\def\csname LT2\endcsname{\color[rgb]{0,0,1}}%
      \expandafter\def\csname LT3\endcsname{\color[rgb]{1,0,1}}%
      \expandafter\def\csname LT4\endcsname{\color[rgb]{0,1,1}}%
      \expandafter\def\csname LT5\endcsname{\color[rgb]{1,1,0}}%
      \expandafter\def\csname LT6\endcsname{\color[rgb]{0,0,0}}%
      \expandafter\def\csname LT7\endcsname{\color[rgb]{1,0.3,0}}%
      \expandafter\def\csname LT8\endcsname{\color[rgb]{0.5,0.5,0.5}}%
    \else
      % gray
      \def\colorrgb#1{\color{black}}%
      \def\colorgray#1{\color[gray]{#1}}%
      \expandafter\def\csname LTw\endcsname{\color{white}}%
      \expandafter\def\csname LTb\endcsname{\color{black}}%
      \expandafter\def\csname LTa\endcsname{\color{black}}%
      \expandafter\def\csname LT0\endcsname{\color{black}}%
      \expandafter\def\csname LT1\endcsname{\color{black}}%
      \expandafter\def\csname LT2\endcsname{\color{black}}%
      \expandafter\def\csname LT3\endcsname{\color{black}}%
      \expandafter\def\csname LT4\endcsname{\color{black}}%
      \expandafter\def\csname LT5\endcsname{\color{black}}%
      \expandafter\def\csname LT6\endcsname{\color{black}}%
      \expandafter\def\csname LT7\endcsname{\color{black}}%
      \expandafter\def\csname LT8\endcsname{\color{black}}%
    \fi
  \fi
    \setlength{\unitlength}{0.0500bp}%
    \ifx\gptboxheight\undefined%
      \newlength{\gptboxheight}%
      \newlength{\gptboxwidth}%
      \newsavebox{\gptboxtext}%
    \fi%
    \setlength{\fboxrule}{0.5pt}%
    \setlength{\fboxsep}{1pt}%
\begin{picture}(7200.00,3240.00)%
    \gplgaddtomacro\gplbacktext{%
      \csname LTb\endcsname%%
      \put(739,704){\makebox(0,0)[r]{\strut{}$-4$}}%
      \csname LTb\endcsname%%
      \put(739,1283){\makebox(0,0)[r]{\strut{}$-3$}}%
      \csname LTb\endcsname%%
      \put(739,1862){\makebox(0,0)[r]{\strut{}$-2$}}%
      \csname LTb\endcsname%%
      \put(739,2440){\makebox(0,0)[r]{\strut{}$-1$}}%
      \csname LTb\endcsname%%
      \put(739,3019){\makebox(0,0)[r]{\strut{}$0$}}%
      \csname LTb\endcsname%%
      \put(871,484){\makebox(0,0){\strut{}$0$}}%
      \csname LTb\endcsname%%
      \put(2028,484){\makebox(0,0){\strut{}$2$}}%
      \csname LTb\endcsname%%
      \put(3186,484){\makebox(0,0){\strut{}$4$}}%
      \csname LTb\endcsname%%
      \put(4343,484){\makebox(0,0){\strut{}$6$}}%
      \csname LTb\endcsname%%
      \put(5501,484){\makebox(0,0){\strut{}$8$}}%
      \csname LTb\endcsname%%
      \put(6658,484){\makebox(0,0){\strut{}$10$}}%
    }%
    \gplgaddtomacro\gplfronttext{%
      \csname LTb\endcsname%%
      \put(343,1861){\makebox(0,0){\strut{}$y$}}%
      \put(3764,154){\makebox(0,0){\strut{}$x$}}%
      \csname LTb\endcsname%%
      \put(3997,2846){\makebox(0,0)[r]{\strut{}Ten Beams}}%
      \csname LTb\endcsname%%
      \put(3997,2626){\makebox(0,0)[r]{\strut{}Continuous}}%
    }%
    \gplbacktext
    \put(0,0){\includegraphics{Beam11}}%
    \gplfronttext
  \end{picture}%
\endgroup

		\caption{$n = 10$.}
		\label{fig:beam10}	
	\end{subfigure}
	\caption{Solution of the discrete catenary problem.}
	\label{fig:catenary}
\end{figure*}

The solutions from KNITRO for two values of $n$ can be found in Figure \ref{fig:catenary}. We shall compare our results to the solution of the continuous case, given by
\begin{align}
	f(x) = a \left( \cosh \left( \frac{2x - x_n}{2a} \right) - \cosh \left( \frac{x_n}{2a} \right) \right),
\end{align}
where $a$ is the positive solution to
\begin{align}
	\frac{x_n}{2a} &= \gamma \sinh \left( \frac{x_n}{2a} \right).
\end{align}
We see that for $n = 4$ (Figure \ref{fig:beam4}) the discrete solution closely approximates the continuous solution. Moreover, it is evident from $n = 10$ (Figure \ref{fig:beam10}) that as $n \rightarrow \infty$ the continuous solution will be obtained. This observation is justified by examining the $L^2$ norm of the difference of the two solutions for a range of number of beams---the result of this can be seen in Figure \ref{fig:convergence}. We find that the discrete solution converges to the continuous solution approximately quadratically. However, when $n \gtrsim 140$ KNITRO is no longer able to solve the optimization problem since the number of variables is approaching $300$.

\begin{figure}[tbp]
	\centering
	% GNUPLOT: LaTeX picture with Postscript
\begingroup
  \makeatletter
  \providecommand\color[2][]{%
    \GenericError{(gnuplot) \space\space\space\@spaces}{%
      Package color not loaded in conjunction with
      terminal option `colourtext'%
    }{See the gnuplot documentation for explanation.%
    }{Either use 'blacktext' in gnuplot or load the package
      color.sty in LaTeX.}%
    \renewcommand\color[2][]{}%
  }%
  \providecommand\includegraphics[2][]{%
    \GenericError{(gnuplot) \space\space\space\@spaces}{%
      Package graphicx or graphics not loaded%
    }{See the gnuplot documentation for explanation.%
    }{The gnuplot epslatex terminal needs graphicx.sty or graphics.sty.}%
    \renewcommand\includegraphics[2][]{}%
  }%
  \providecommand\rotatebox[2]{#2}%
  \@ifundefined{ifGPcolor}{%
    \newif\ifGPcolor
    \GPcolortrue
  }{}%
  \@ifundefined{ifGPblacktext}{%
    \newif\ifGPblacktext
    \GPblacktexttrue
  }{}%
  % define a \g@addto@macro without @ in the name:
  \let\gplgaddtomacro\g@addto@macro
  % define empty templates for all commands taking text:
  \gdef\gplbacktext{}%
  \gdef\gplfronttext{}%
  \makeatother
  \ifGPblacktext
    % no textcolor at all
    \def\colorrgb#1{}%
    \def\colorgray#1{}%
  \else
    % gray or color?
    \ifGPcolor
      \def\colorrgb#1{\color[rgb]{#1}}%
      \def\colorgray#1{\color[gray]{#1}}%
      \expandafter\def\csname LTw\endcsname{\color{white}}%
      \expandafter\def\csname LTb\endcsname{\color{black}}%
      \expandafter\def\csname LTa\endcsname{\color{black}}%
      \expandafter\def\csname LT0\endcsname{\color[rgb]{1,0,0}}%
      \expandafter\def\csname LT1\endcsname{\color[rgb]{0,1,0}}%
      \expandafter\def\csname LT2\endcsname{\color[rgb]{0,0,1}}%
      \expandafter\def\csname LT3\endcsname{\color[rgb]{1,0,1}}%
      \expandafter\def\csname LT4\endcsname{\color[rgb]{0,1,1}}%
      \expandafter\def\csname LT5\endcsname{\color[rgb]{1,1,0}}%
      \expandafter\def\csname LT6\endcsname{\color[rgb]{0,0,0}}%
      \expandafter\def\csname LT7\endcsname{\color[rgb]{1,0.3,0}}%
      \expandafter\def\csname LT8\endcsname{\color[rgb]{0.5,0.5,0.5}}%
    \else
      % gray
      \def\colorrgb#1{\color{black}}%
      \def\colorgray#1{\color[gray]{#1}}%
      \expandafter\def\csname LTw\endcsname{\color{white}}%
      \expandafter\def\csname LTb\endcsname{\color{black}}%
      \expandafter\def\csname LTa\endcsname{\color{black}}%
      \expandafter\def\csname LT0\endcsname{\color{black}}%
      \expandafter\def\csname LT1\endcsname{\color{black}}%
      \expandafter\def\csname LT2\endcsname{\color{black}}%
      \expandafter\def\csname LT3\endcsname{\color{black}}%
      \expandafter\def\csname LT4\endcsname{\color{black}}%
      \expandafter\def\csname LT5\endcsname{\color{black}}%
      \expandafter\def\csname LT6\endcsname{\color{black}}%
      \expandafter\def\csname LT7\endcsname{\color{black}}%
      \expandafter\def\csname LT8\endcsname{\color{black}}%
    \fi
  \fi
    \setlength{\unitlength}{0.0500bp}%
    \ifx\gptboxheight\undefined%
      \newlength{\gptboxheight}%
      \newlength{\gptboxwidth}%
      \newsavebox{\gptboxtext}%
    \fi%
    \setlength{\fboxrule}{0.5pt}%
    \setlength{\fboxsep}{1pt}%
\begin{picture}(7200.00,3240.00)%
    \gplgaddtomacro\gplbacktext{%
      \csname LTb\endcsname%%
      \put(946,704){\makebox(0,0)[r]{\strut{}$10^{-4}$}}%
      \csname LTb\endcsname%%
      \put(946,1167){\makebox(0,0)[r]{\strut{}$10^{-3}$}}%
      \csname LTb\endcsname%%
      \put(946,1630){\makebox(0,0)[r]{\strut{}$10^{-2}$}}%
      \csname LTb\endcsname%%
      \put(946,2093){\makebox(0,0)[r]{\strut{}$10^{-1}$}}%
      \csname LTb\endcsname%%
      \put(946,2556){\makebox(0,0)[r]{\strut{}$10^{0}$}}%
      \csname LTb\endcsname%%
      \put(946,3019){\makebox(0,0)[r]{\strut{}$10^{1}$}}%
      \csname LTb\endcsname%%
      \put(1078,484){\makebox(0,0){\strut{}$10^{0}$}}%
      \csname LTb\endcsname%%
      \put(3746,484){\makebox(0,0){\strut{}$10^{1}$}}%
      \csname LTb\endcsname%%
      \put(6413,484){\makebox(0,0){\strut{}$10^{2}$}}%
    }%
    \gplgaddtomacro\gplfronttext{%
      \csname LTb\endcsname%%
      \put(198,1861){\rotatebox{-270}{\makebox(0,0){\strut{}Error}}}%
      \put(3940,154){\makebox(0,0){\strut{}$n$}}%
      \csname LTb\endcsname%%
      \put(2200,1097){\makebox(0,0)[r]{\strut{}Data}}%
      \csname LTb\endcsname%%
      \put(2200,877){\makebox(0,0)[r]{\strut{}$5.92 n^{-1.99}$}}%
    }%
    \gplbacktext
    \put(0,0){\includegraphics{Convergence}}%
    \gplfronttext
  \end{picture}%
\endgroup

	\caption{The $L^2$ norm of the difference between the discrete solution and the continuous solution.}
	\label{fig:convergence}
\end{figure}

%\bibliography{Ref}

\end{document}
