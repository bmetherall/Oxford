\documentclass[11pt,a4paper,twocolumn]{article}

\usepackage{amsfonts}
\usepackage{amsmath}
\usepackage{geometry}
\usepackage{xcolor}
\usepackage{graphicx}
%\usepackage[subfolder,cleanup]{gnuplottex}
%\usepackage{amsthm}
%\usepackage{enumitem}
%\usepackage{wrapfig}
\usepackage{subcaption}
%\usepackage{hyperref}
%\usepackage{tikz}

\usepackage{lipsum}

% Nicer brackets for operators
\let\originalleft\left
\let\originalright\right
\renewcommand{\left}{\mathopen{}\mathclose\bgroup\originalleft}
\renewcommand{\right}{\aftergroup\egroup\originalright}

% Math operators
\providecommand{\bigO}[1]{\ensuremath{\mathop{}\mathopen{}\mathcal{O}\mathopen{}\left(#1\right)}}

% Macros
\newcommand{\diff}[3][\hspace{-0.5pt}]{\frac{\textrm{d}^{#1}#2}{\textrm{d}{#3}^{#1}}}
\newcommand{\pdiff}[3][\hspace{-0.5pt}]{\frac{\partial^{#1}#2}{\partial{#3}^{#1}}}
\newcommand{\df}{\, \textrm{d}}
%\newcommand{\eps}{\varepsilon}

% Row colouring in tables
%\usepackage[table]{xcolor}
%\rowcolors{2}{gray!25}{white}

% Margin size
\newgeometry{margin=2cm}

% Reference style
%\bibliographystyle{ieeetr}

\title{Continuous Optimization \\ InFoMM Assignment}
\author{Brady Metherall}
\date{16 March, 2020}

\begin{document}
\maketitle

unconstrained rosenbrock

quadratic constrained rosenbrock wikipedia

catenary problem

different n

limit as n goes to inf goes to continuous solution

\begin{align}
	\min_{x, y \in \mathbb{R}} 100(y - x^2)^2 + (1 - x)^2
\end{align}

\begin{align}
	(x - 1)^2 + y &\geq 1 \\
	x + y &\leq 2
\end{align}

n+1 points, n beams

\begin{align}
	\min_{\mathbf{x}, \mathbf{y} \in \mathbb{R}^{n+1}} m g \left( \frac{1}{2} y_0 + \frac{1}{2} y_n + \sum_{i = 1}^{n-1} y_i \right)
\end{align}

\begin{align}
	x_0 &= 0, & x_n &= \gamma n L, \\
	y_0 &=0, & y_n &= 0,
\end{align}

\begin{align}
	(x_i - x_{i+1})^2 + (y_i - y_{i+1})^2 = L^2
\end{align}

\begin{align}
	x_n &= 10, & g &= 1, & m &= 1, \\
	\gamma &= \frac{4}{5}, & L &= \frac{x_n}{\gamma n}.
\end{align}


\begin{figure}[tbp]
	\centering
	\input{Rosen}
	\caption{}
	\label{fig:rosen}
\end{figure}

\lipsum[1-8]

\begin{table*}
	\centering
	\begin{tabular}{lcccccc}
		\hline\noalign{\smallskip}
		Starting value & $(-2,-3)$ & $(2,1)$ & $(-3,4)$ & $(\pi,-\textrm{e})$ & $(0,0)$ & $(2,4)$ \\
		\hline\noalign{\smallskip}
		No derivatives & 36 & 16 & 40 & 22 & 17 & 18 \\
		Gradient & 36 & 16 & 41 & 22 &  & \\
		Hessian & 20 & 12 & 23 & 12 & & \\
		\noalign{\smallskip}\hline\noalign{\smallskip}
	\end{tabular}
	\caption{}
	\label{tab:iterations}
\end{table*}



\begin{figure}[tbp]
	\centering
	\input{RosenConst}
	\caption{}
	\label{fig:rosenconst}
\end{figure}

\begin{figure*}
	\centering
	\begin{subfigure}{\textwidth}
		\centering
		\input{Beam5}
		\caption{}
		\label{fig:beam5}
	\end{subfigure} \\
	\begin{subfigure}{\textwidth}
		\centering
		\input{Beam11}
		\caption{}
		\label{fig:beam11}	
	\end{subfigure}
	\caption{}
	\label{fig:catenary}
\end{figure*}


\begin{figure}[tbp]
	\centering
	\input{Convergence}
	\caption{}
	\label{fig:convergence}
\end{figure}

%\bibliography{Ref}


\end{document}
